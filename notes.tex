\documentclass[12pt,english,oneside]{scrbook}
\usepackage{ae,aecompl}
\usepackage[T1]{fontenc}
\usepackage[latin9]{inputenc}
\usepackage{geometry}
%\geometry{verbose,tmargin=1in,bmargin=1in,lmargin=1in,rmargin=1in}
\setlength{\parskip}{\medskipamount}
\setlength{\parindent}{0pt}
\usepackage{amsmath}
\usepackage{amsthm}
\usepackage{amssymb}
\usepackage{setspace}
\setstretch{1.1}

\makeatletter
\numberwithin{equation}{chapter}
\theoremstyle{definition}
\newtheorem*{example*}{\protect\examplename}
\theoremstyle{plain}
\newtheorem{thm}{\protect\theoremname}[section]

\@ifundefined{date}{}{\date{}}
\AtBeginDocument{
\addtolength{\abovedisplayskip}{0ex}
\addtolength{\abovedisplayshortskip}{0ex}
\addtolength{\belowdisplayskip}{0ex}
\addtolength{\belowdisplayshortskip}{0ex}
}

\newtheorem{theorem}{Theorem}[section]
\newtheorem{lemma}{Lemma}[section]
\newtheorem*{corollary*}{Corollary}

\makeatother

\usepackage{babel}
\providecommand{\examplename}{Example}
\providecommand{\theoremname}{Theorem}

\begin{document}


\title{Measure Theory: Cheatsheet}
\author{Ronald Mangang}

\maketitle
\tableofcontents

\thispagestyle{empty}
\newpage
\pagenumbering{arabic}


\chapter{Measure Spaces}

\paragraph{Generation of $\sigma$-algebras.}

Let $\Omega$ be a nonempty set. Let $\mathcal{A}\subset 2^\Omega$. The smallest $\sigma$-algebra containing $\mathcal{A}$ is called the $\sigma$-algebra generated by $\mathcal{A}$. We denote it by $\sigma(\mathcal{A})$. Trivially, $\sigma(\mathcal{A})\subset 2^\Omega$.

Similar concepts are defined similarly.

Let $X$ be a metric space. The $\sigma$-algbra generated by the open sets (or balls) is called Borel $\sigma$-algebra. The sets that belong to Borel $\sigma$-algebra are called Borel sets. All open or closed sets are then Borel sets.

It is important to know that sets of the form $(a,\infty)$ generate all Borel sets on $\mathbb{R}$.

\paragraph{Limit superior and limit inferior.}

Let $\{A_n\}_{n\geq 0}$ be a sequence of sets. Then
\begin{equation}
  \lim \sup A_n = \cap_{n=1}^\infty \cup_{k\geq n} A_k, \hspace{2em}
  \lim \inf A_n = \cup_{n=1}^\infty \cap_{k\geq n} A_k
\end{equation}

From the definitions, we derive
\begin{equation}
  \lim \inf A_n \leq \lim \sup A_n
\end{equation}

The sequence ${A_n}$ converges to $A$ iff $\lim \inf A_n = \lim \sup A_n$.

\paragraph{Monotone class theorem.}

Let $\mathcal{A}$ be an algebra. Let $\mathcal{F}$ and $\mathcal{M}$ be $\sigma$-algebra and monotone class generated by $\mathcal{A}$ respectively. Then $\mathcal{F} = \mathcal{M}$.

Proof is easy. $\mathcal{F}$ is a montone class by definition. This implies that $\mathcal{M}\subset \mathcal{F}$. A simple use of logic will imply $\mathcal{M}$ is also a $\sigma$-algebra which means $\mathcal{F} \subset \mathcal{M}$.

\paragraph{Measurable space vs. measure space.}

Any set $\Omega$ bundled with a $\sigma$-algebra $\mathcal{F}$ defined on $\Omega$ makes a measurable space. If we define a measure $\mu : \mathcal{F} \to [0,\infty]$ and associate it to the measurable space, it becomes a measure space.

In short, $(\Omega, \mathcal{F})$ is a measurable space and $(\Omega, \mathcal{F}, \mu)$ is a measure space.

\paragraph{Lebesgue measure.}

Consider the measurable space $(\mathbb{R}^n, \mathcal{B}^n)$ where $\mathcal{B}^n$ denote the Borel $\sigma$-algebra on $\mathbb{R}^n$. We define Lebesgue measure $l: \mathcal{B}^n \to [0,\infty)$ by
\begin{equation}
  l((a_1,b_1)\times(a_2,b_2)\times\cdots\times(a_n,b_n)) = \prod_{i=1}^n (b_i-a_i).
\end{equation}
The value of the measure remains the same if the open intervals are replaced with closed intervals. Lebesgue measure generalizes the sense of length, area, volume.

\chapter{Measurability}

\paragraph{Checking measurability}

Let $(\Omega, \mathcal{F})$ and $(\Omega^\prime, \mathcal{F}^\prime)$ be two measurable spaces. A function $f:\Omega \to \Omega^\prime$ is measurable if for any $A\in \mathcal{F}^\prime$, we have $f^{-1}(A) \in \mathcal{F}$.

Informally, a measurable function is one whose inverse images of measurable sets are also measurable. That is, it preserves the structure of a measurable space. In that regard, they are like continuous functions which preserve the structure of topological spaces, i.e., the inverse image of an open set is open.

\paragraph{Measurability on a generator.}

To check if a set function $f:(\Omega, \mathcal{F})\to(\Omega^\prime, \mathcal{F}^\prime)$ is $\mathcal{F}$-$\mathcal{F}^\prime$ measurable it is sufficient
to check that $f$ is measurable on a set $B\subset \mathcal{F}^\prime$ where $B$ generates $\mathcal{F}^\prime$.

Since intervals of the form $(-\infty, a)$ generate the Borel $\sigma$-algebra on $\mathbb{R}$ we sometimes say $f$ is measurable if $\{x: f(x)<a, a\in \mathbb{R} \}$ is measurable.

\paragraph{$\sigma$-algebras from measurable functions.}

$f^{-1}(\mathcal{F}^\prime)$ is the smallest $\sigma$-algebra for $f$ to be measurable.

We can extend this idea for a collection of functions. Suppose $(\sigma_i,
\mathcal{F}_i)$ for some index $i$ are measurable spaces. The smallest $\sigma$-algebra so that $f_i : \sigma \to \sigma_i$ are all measurable is then given by
\begin{equation}
  \sigma(f_i) = \sigma (\cup f_i^{-1}(\mathcal{F}_i))
\end{equation}

\paragraph{Operations on measurable functions.}

Sum, difference, product, scalar product, composition on measurable functions yield measurable functions.

Let $f_i$ be measurable functions. Then $\sup f_i$, $\inf f_i$, $\lim \sup f_i$ and $\lim \inf f_i$ are measurable.

\paragraph{The concept of almost everywhere.}

Informally, we say that a property $P$ holds almost everywhere if the set of elements for which $P$ does not hold has measure zero. In Euclidean spaces, any countable set has measure zero --- this implies that any property which holds for all points in $\mathbb{R}^n$ except for countably many points is said to hold almost everywhere.

The mathematical concept of ``almost surely'' is closely related to this.

\paragraph{Approximation by simple functions.}

Any nonnegative measurable function $f$ can be written as
\begin{equation}
  f(x) = \lim_{n\to \infty} \sum_{i=0}^n a_i 1_{E_i}(x)
\end{equation}
where $1_{E_i}$ is the indicator function of the measurable set $E_i$. In essence, we are approximating $f$ by an increasing sequence of nonnegative simple functions defined by $\sum_{i=0}^n a_i 1_{E_i}(x)$.

\chapter{Lebesgue Integration}

\paragraph{Simple functions.}

From the previous chapter, a simple function is of the form
\begin{equation}
  s = \sum_{i=0}^n a_i 1_{E_i}(x)
\end{equation}

Then the Lebesgue integral of $s$ with respect to measure $\mu$ is
\begin{equation}
  \int s d\mu = \sum_{i=0}^n a_i \mu (E_i)
\end{equation}
The integral $\int s d\mu$ does not depend on the representation of $s$.

\paragraph{Measurable functions.}

From the previous chapter, we know how to approximate a nonnegative measurable function by an increasing sequence of nonnegative simple functions. Let
\begin{equation}
  f(x) = \lim_{n\to \infty} \sum_{i=0}^n a_i 1_{E_i}(x)
\end{equation}

Then the Lebesgue integral of $f$ with respect to measure $\mu$ is
\begin{equation}
  \int f d\mu = \sup \left( \sum_{i=0}^n a_i (\mu (E_i) \right)
\end{equation}

\paragraph{``Almost everywhere" revisted.}

From the way how a Lebesgue integral is defined, it follows ``almost trivially'' that if $f\leq g$ almost everywhere then $\int f d\mu \leq \int g d\mu$. Also if $f=g$ almost everywhere then $\int f d\mu = \int g d\mu$.

Suppose $f=0$ almost everywhere. Then $\int f d\mu = 0$. The converse is also true.

Well the take-home message is: in the grand scheme of things, irregularities in countably many cases do not make a difference. Well sort of.

Lebesgue integrals are in many ways like the ordinary Riemann integrals. They obey linearity and monotone convergence.

\paragraph{Monotone convergence.}

Let $\{f_n\}$ be such that $\lim_{n\to \infty} f_n = f$. Then
\begin{equation}
  \int f d\mu = \lim_{n\to\infty} \int f_n d\mu.
\end{equation}

This is what helped us to defined Lebesgue integrals for nonnegative measurable functions using a sequence of increasing simple functions. A careful look reveals that $\{f_n\}$ are the simple functions which approximate $f$.





\end{document}



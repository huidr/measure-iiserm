\documentclass{amsart}
% COMMANDS

\newcommand{\eps}{\epsilon}
\newcommand{\veps}{\varepsilon}
\newcommand{\Qed}{\begin{flushright}\qed\end{flushright}}
\newcommand{\parinn}{\setlength{\parindent}{1cm}}
\newcommand{\parinf}{\setlength{\parindent}{0cm}}
\newcommand{\norm}{\|\cdot\|}
\newcommand{\inorm}{\norm_{\infty}}
\newcommand{\opensets}{\{V_{\alpha}\}_{\alpha\in I}}
\newcommand{\oset}{V_{\alpha}}
\newcommand{\opset}[1]{V_{\alpha_{#1}}}
\newcommand{\lub}{\text{lub}}
\newcommand{\del}[2]{\frac{\partial #1}{\partial #2}}
\newcommand{\Del}[3]{\frac{\partial^{#1} #2}{\partial^{#1} #3}}
\newcommand{\deld}[2]{\dfrac{\partial #1}{\partial #2}}
\newcommand{\Deld}[3]{\dfrac{\partial^{#1} #2}{\partial^{#1} #3}}
\newcommand{\lm}{\lambda}
\newcommand{\uin}{\mathbin{\rotatebox[origin=c]{90}{$\in$}}}
\newcommand{\usubset}{\mathbin{\rotatebox[origin=c]{90}{$\subset$}}}
\newcommand{\lt}{\left}
\newcommand{\rt}{\right}
\newcommand{\bs}[1]{\boldsymbol{#1}}
\newcommand{\exs}{\exists}
\newcommand{\st}{\strut}
\newcommand{\dps}[1]{\displaystyle{#1}}

\newcommand{\sol}{\setlength{\parindent}{0cm}\textbf{\textit{Solution:}}\setlength{\parindent}{1cm} }
\newcommand{\solve}[1]{\setlength{\parindent}{0cm}\textbf{\textit{Solution: }}\setlength{\parindent}{1cm}#1 \Qed}

%---------------------------------------
% BlackBoard Math Fonts :-
%---------------------------------------

%Captital Letters
\newcommand{\bbA}{\mathbb{A}}	\newcommand{\bbB}{\mathbb{B}}
\newcommand{\bbC}{\mathbb{C}}	\newcommand{\bbD}{\mathbb{D}}
\newcommand{\bbE}{\mathbb{E}}	\newcommand{\bbF}{\mathbb{F}}
\newcommand{\bbG}{\mathbb{G}}	\newcommand{\bbH}{\mathbb{H}}
\newcommand{\bbI}{\mathbb{I}}	\newcommand{\bbJ}{\mathbb{J}}
\newcommand{\bbK}{\mathbb{K}}	\newcommand{\bbL}{\mathbb{L}}
\newcommand{\bbM}{\mathbb{M}}	\newcommand{\bbN}{\mathbb{N}}
\newcommand{\bbO}{\mathbb{O}}	\newcommand{\bbP}{\mathbb{P}}
\newcommand{\bbQ}{\mathbb{Q}}	\newcommand{\bbR}{\mathbb{R}}
\newcommand{\bbS}{\mathbb{S}}	\newcommand{\bbT}{\mathbb{T}}
\newcommand{\bbU}{\mathbb{U}}	\newcommand{\bbV}{\mathbb{V}}
\newcommand{\bbW}{\mathbb{W}}	\newcommand{\bbX}{\mathbb{X}}
\newcommand{\bbY}{\mathbb{Y}}	\newcommand{\bbZ}{\mathbb{Z}}

%---------------------------------------
% MathCal Fonts :-
%---------------------------------------

%Captital Letters
\newcommand{\mcA}{\mathcal{A}}	\newcommand{\mcB}{\mathcal{B}}
\newcommand{\mcC}{\mathcal{C}}	\newcommand{\mcD}{\mathcal{D}}
\newcommand{\mcE}{\mathcal{E}}	\newcommand{\mcF}{\mathcal{F}}
\newcommand{\mcG}{\mathcal{G}}	\newcommand{\mcH}{\mathcal{H}}
\newcommand{\mcI}{\mathcal{I}}	\newcommand{\mcJ}{\mathcal{J}}
\newcommand{\mcK}{\mathcal{K}}	\newcommand{\mcL}{\mathcal{L}}
\newcommand{\mcM}{\mathcal{M}}	\newcommand{\mcN}{\mathcal{N}}
\newcommand{\mcO}{\mathcal{O}}	\newcommand{\mcP}{\mathcal{P}}
\newcommand{\mcQ}{\mathcal{Q}}	\newcommand{\mcR}{\mathcal{R}}
\newcommand{\mcS}{\mathcal{S}}	\newcommand{\mcT}{\mathcal{T}}
\newcommand{\mcU}{\mathcal{U}}	\newcommand{\mcV}{\mathcal{V}}
\newcommand{\mcW}{\mathcal{W}}	\newcommand{\mcX}{\mathcal{X}}
\newcommand{\mcY}{\mathcal{Y}}	\newcommand{\mcZ}{\mathcal{Z}}



%---------------------------------------
% Bold Math Fonts :-
%---------------------------------------

%Captital Letters
\newcommand{\bmA}{\boldsymbol{A}}	\newcommand{\bmB}{\boldsymbol{B}}
\newcommand{\bmC}{\boldsymbol{C}}	\newcommand{\bmD}{\boldsymbol{D}}
\newcommand{\bmE}{\boldsymbol{E}}	\newcommand{\bmF}{\boldsymbol{F}}
\newcommand{\bmG}{\boldsymbol{G}}	\newcommand{\bmH}{\boldsymbol{H}}
\newcommand{\bmI}{\boldsymbol{I}}	\newcommand{\bmJ}{\boldsymbol{J}}
\newcommand{\bmK}{\boldsymbol{K}}	\newcommand{\bmL}{\boldsymbol{L}}
\newcommand{\bmM}{\boldsymbol{M}}	\newcommand{\bmN}{\boldsymbol{N}}
\newcommand{\bmO}{\boldsymbol{O}}	\newcommand{\bmP}{\boldsymbol{P}}
\newcommand{\bmQ}{\boldsymbol{Q}}	\newcommand{\bmR}{\boldsymbol{R}}
\newcommand{\bmS}{\boldsymbol{S}}	\newcommand{\bmT}{\boldsymbol{T}}
\newcommand{\bmU}{\boldsymbol{U}}	\newcommand{\bmV}{\boldsymbol{V}}
\newcommand{\bmW}{\boldsymbol{W}}	\newcommand{\bmX}{\boldsymbol{X}}
\newcommand{\bmY}{\boldsymbol{Y}}	\newcommand{\bmZ}{\boldsymbol{Z}}
%Small Letters
\newcommand{\bma}{\boldsymbol{a}}	\newcommand{\bmb}{\boldsymbol{b}}
\newcommand{\bmc}{\boldsymbol{c}}	\newcommand{\bmd}{\boldsymbol{d}}
\newcommand{\bme}{\boldsymbol{e}}	\newcommand{\bmf}{\boldsymbol{f}}
\newcommand{\bmg}{\boldsymbol{g}}	\newcommand{\bmh}{\boldsymbol{h}}
\newcommand{\bmi}{\boldsymbol{i}}	\newcommand{\bmj}{\boldsymbol{j}}
\newcommand{\bmk}{\boldsymbol{k}}	\newcommand{\bml}{\boldsymbol{l}}
\newcommand{\bmm}{\boldsymbol{m}}	\newcommand{\bmn}{\boldsymbol{n}}
\newcommand{\bmo}{\boldsymbol{o}}	\newcommand{\bmp}{\boldsymbol{p}}
\newcommand{\bmq}{\boldsymbol{q}}	\newcommand{\bmr}{\boldsymbol{r}}
\newcommand{\bms}{\boldsymbol{s}}	\newcommand{\bmt}{\boldsymbol{t}}
\newcommand{\bmu}{\boldsymbol{u}}	\newcommand{\bmv}{\boldsymbol{v}}
\newcommand{\bmw}{\boldsymbol{w}}	\newcommand{\bmx}{\boldsymbol{x}}
\newcommand{\bmy}{\boldsymbol{y}}	\newcommand{\bmz}{\boldsymbol{z}}

%---------------------------------------
% Scr Math Fonts :-
%---------------------------------------

\newcommand{\sA}{{\mathscr{A}}}   \newcommand{\sB}{{\mathscr{B}}}
\newcommand{\sC}{{\mathscr{C}}}   \newcommand{\sD}{{\mathscr{D}}}
\newcommand{\sE}{{\mathscr{E}}}   \newcommand{\sF}{{\mathscr{F}}}
\newcommand{\sG}{{\mathscr{G}}}   \newcommand{\sH}{{\mathscr{H}}}
\newcommand{\sI}{{\mathscr{I}}}   \newcommand{\sJ}{{\mathscr{J}}}
\newcommand{\sK}{{\mathscr{K}}}   \newcommand{\sL}{{\mathscr{L}}}
\newcommand{\sM}{{\mathscr{M}}}   \newcommand{\sN}{{\mathscr{N}}}
\newcommand{\sO}{{\mathscr{O}}}   \newcommand{\sP}{{\mathscr{P}}}
\newcommand{\sQ}{{\mathscr{Q}}}   \newcommand{\sR}{{\mathscr{R}}}
\newcommand{\sS}{{\mathscr{S}}}   \newcommand{\sT}{{\mathscr{T}}}
\newcommand{\sU}{{\mathscr{U}}}   \newcommand{\sV}{{\mathscr{V}}}
\newcommand{\sW}{{\mathscr{W}}}   \newcommand{\sX}{{\mathscr{X}}}
\newcommand{\sY}{{\mathscr{Y}}}   \newcommand{\sZ}{{\mathscr{Z}}}


%---------------------------------------
% Math Fraktur Font
%---------------------------------------

%Captital Letters
\newcommand{\mfA}{\mathfrak{A}}	\newcommand{\mfB}{\mathfrak{B}}
\newcommand{\mfC}{\mathfrak{C}}	\newcommand{\mfD}{\mathfrak{D}}
\newcommand{\mfE}{\mathfrak{E}}	\newcommand{\mfF}{\mathfrak{F}}
\newcommand{\mfG}{\mathfrak{G}}	\newcommand{\mfH}{\mathfrak{H}}
\newcommand{\mfI}{\mathfrak{I}}	\newcommand{\mfJ}{\mathfrak{J}}
\newcommand{\mfK}{\mathfrak{K}}	\newcommand{\mfL}{\mathfrak{L}}
\newcommand{\mfM}{\mathfrak{M}}	\newcommand{\mfN}{\mathfrak{N}}
\newcommand{\mfO}{\mathfrak{O}}	\newcommand{\mfP}{\mathfrak{P}}
\newcommand{\mfQ}{\mathfrak{Q}}	\newcommand{\mfR}{\mathfrak{R}}
\newcommand{\mfS}{\mathfrak{S}}	\newcommand{\mfT}{\mathfrak{T}}
\newcommand{\mfU}{\mathfrak{U}}	\newcommand{\mfV}{\mathfrak{V}}
\newcommand{\mfW}{\mathfrak{W}}	\newcommand{\mfX}{\mathfrak{X}}
\newcommand{\mfY}{\mathfrak{Y}}	\newcommand{\mfZ}{\mathfrak{Z}}
%Small Letters
\newcommand{\mfa}{\mathfrak{a}}	\newcommand{\mfb}{\mathfrak{b}}
\newcommand{\mfc}{\mathfrak{c}}	\newcommand{\mfd}{\mathfrak{d}}
\newcommand{\mfe}{\mathfrak{e}}	\newcommand{\mff}{\mathfrak{f}}
\newcommand{\mfg}{\mathfrak{g}}	\newcommand{\mfh}{\mathfrak{h}}
\newcommand{\mfi}{\mathfrak{i}}	\newcommand{\mfj}{\mathfrak{j}}
\newcommand{\mfk}{\mathfrak{k}}	\newcommand{\mfl}{\mathfrak{l}}
\newcommand{\mfm}{\mathfrak{m}}	\newcommand{\mfn}{\mathfrak{n}}
\newcommand{\mfo}{\mathfrak{o}}	\newcommand{\mfp}{\mathfrak{p}}
\newcommand{\mfq}{\mathfrak{q}}	\newcommand{\mfr}{\mathfrak{r}}
\newcommand{\mfs}{\mathfrak{s}}	\newcommand{\mft}{\mathfrak{t}}
\newcommand{\mfu}{\mathfrak{u}}	\newcommand{\mfv}{\mathfrak{v}}
\newcommand{\mfw}{\mathfrak{w}}	\newcommand{\mfx}{\mathfrak{x}}
\newcommand{\mfy}{\mathfrak{y}}	\newcommand{\mfz}{\mathfrak{z}}

%%%%%%%%%%%%%%%%%%%%%%%%%%%%%%%%%
% PACKAGES
%%%%%%%%%%%%%%%%%%%%%%%%%%%%%%%%%

\usepackage[tmargin=2cm,rmargin=1in,lmargin=1in,margin=0.85in,bmargin=2cm,footskip=.2in]{geometry}
\usepackage{amsmath,amsfonts,amsthm,amssymb,mathtools}
\usepackage{bookmark}
\usepackage{enumitem}
\setlist[itemize]{noitemsep, topsep=1pt}
\setlist[enumerate]{noitemsep, topsep=1pt}
\usepackage{hyperref,theoremref}
\hypersetup{
	pdftitle={Theory of Computation},
	colorlinks=true, linkcolor=purple!80,
	bookmarksnumbered=true,
	bookmarksopen=true
}
\usepackage[most,many,breakable]{tcolorbox}
\usepackage{xcolor}
\usepackage{varwidth}
\usepackage{varwidth}
\usepackage{etoolbox}
%\usepackage{authblk}
\usepackage{nameref}
\usepackage{multicol,array}
\usepackage{afterpage}
\usepackage{tikz}
\usetikzlibrary{fadings}
\usepackage{tikz-cd}
\usepackage{pdfpages}
\usepackage{titlesec}

%\usepackage{import}
%\usepackage{xifthen}
%\usepackage{transparent}

%%%%%%%%%%%%%%%%%%%%%%%%%%%%%%
% COLORS
%%%%%%%%%%%%%%%%%%%%%%%%%%%%%%

\definecolor{myg}{RGB}{56, 140, 70}
\definecolor{myb}{RGB}{45, 111, 177}
\definecolor{myr}{RGB}{199, 68, 64}
\definecolor{mytheorembg}{HTML}{F2F2F9}
\definecolor{mytheoremfr}{HTML}{00007B}
\definecolor{myexamplebg}{HTML}{F2FBF8}
\definecolor{myexamplefr}{HTML}{88D6D1}
\definecolor{myexampleti}{HTML}{2A7F7F}
\definecolor{mydefinitbg}{HTML}{E5E5FF}
\definecolor{mydefinitfr}{HTML}{3F3FA3}
\definecolor{notesgreen}{RGB}{0,162,0}
\definecolor{myp}{RGB}{197, 92, 212}
\definecolor{mygr}{HTML}{2C3338}
\definecolor{myred}{RGB}{127,0,0}
\definecolor{myyellow}{RGB}{169,121,69}

\setlength{\parindent}{0cm}   
\setlength{\parskip}{6pt}  

%================================
% THEOREM BOX
%================================

\tcbuselibrary{theorems,skins,hooks}
\newtcbtheorem[number within=section]{Theorem}{Theorem}
{%
	enhanced,
	breakable,
	colback = mytheorembg,
	frame hidden,
	boxrule = 1sp,
	borderline west = {2pt}{0pt}{mytheoremfr},
	sharp corners,
	detach title,
	before upper = \tcbtitle\par\smallskip,
	coltitle = mytheoremfr,
	fonttitle = \bfseries\sffamily,
	description font = \mdseries,
	separator sign none,
	segmentation style={solid, mytheoremfr},
        description delimiters parenthesis
}
{th}

%================================
% Corollary
%================================
\tcbuselibrary{theorems,skins,hooks}
\newtcbtheorem[number within=section]{corolary}{Corollary}
{%
	enhanced
	,breakable
	,colback = myp!10
	,frame hidden
	,boxrule = 0sp
	,borderline west = {2pt}{0pt}{myp!85!black}
	,sharp corners
	,detach title
	,before upper = \tcbtitle\par\smallskip
	,coltitle = myp!85!black
	,fonttitle = \bfseries\sffamily
	,description font = \mdseries
	,separator sign none
	,segmentation style={solid, myp!85!black}
        ,description delimiters parenthesis
}
{th}

%================================
% LEMMA
%================================

\tcbuselibrary{theorems,skins,hooks}
\newtcbtheorem[number within=section]{lemma}{Lemma}
{%
	enhanced
	,breakable
	,colback = myg!10
	,frame hidden
	,boxrule = 0sp
	,borderline west = {2pt}{0pt}{myg}
	,sharp corners
	,detach title
	,before upper = \tcbtitle\par\smallskip
	,coltitle = myg!85!black
	,fonttitle = \bfseries\sffamily
	,description font = \mdseries
	,separator sign none
	,segmentation style={solid, myg!85!black}
        ,description delimiters parenthesis
}
{th}

%================================
% EXAMPLE BOX
%================================

\newtcbtheorem[number within=section]{Example}{Example}
{%
	colback = myexamplebg
	,breakable
	,colframe = myexamplefr
	,coltitle = myexampleti
	,boxrule = 1pt
	,sharp corners
	,detach title
	,before upper=\tcbtitle\par\smallskip
	,fonttitle = \bfseries
	,description font = \mdseries
	,separator sign none
	,description delimiters parenthesis
}
{ex}

%================================
% REMARK BOX
%================================

\tcbuselibrary{theorems,skins,hooks}
\newtcbtheorem[number within=section]{remark}{Remark}
{%
	enhanced
	,breakable
	,colback = myyellow!10
	,frame hidden
	,boxrule = 0sp
	,borderline west = {2pt}{0pt}{myyellow!85!black}
	,sharp corners
	,detach title
	,before upper = \tcbtitle\par\smallskip
	,coltitle = myyellow!85!black
	,fonttitle = \bfseries\sffamily
	,description font = \mdseries
	,separator sign none
	,segmentation style={solid, myyellow!85!black}
        ,description delimiters parenthesis
}
{th}

%================================
% DEFINITION BOX
%================================

\newtcbtheorem[number within=section]{Definition}{Definition}{enhanced,
	before skip=4mm,after skip=4mm, colback=green!05,colframe=green!80!black, boxrule=0.1mm,
	attach boxed title to top left={xshift=.5cm,yshift*=1mm-\tcboxedtitleheight}, varwidth boxed title*=-3cm,
	boxed title style={frame code={
					\path[fill=tcbcolback]
					([yshift=-1mm,xshift=-1mm]frame.north west)
					arc[start angle=0,end angle=180,radius=1mm]
					([yshift=-1mm,xshift=1mm]frame.north east)
					arc[start angle=180,end angle=0,radius=1mm];
					\path[left color=tcbcolback!60!black,right color=tcbcolback!60!black,
						middle color=tcbcolback!80!black]
					([xshift=-2mm]frame.north west) -- ([xshift=2mm]frame.north east)
					[rounded corners=1mm]-- ([xshift=1mm,yshift=-1mm]frame.north east)
					-- (frame.south east) -- (frame.south west)
					-- ([xshift=-1mm,yshift=-1mm]frame.north west)
					[sharp corners]-- cycle;
				},interior engine=empty,
		},
	fonttitle=\bfseries,
	title={#2},#1}{def}
\newtcbtheorem[number within=chapter]{definition}{Definition}{enhanced,
	before skip=2mm,after skip=2mm, colback=red!5,colframe=red!80!black,boxrule=0.5mm,
	attach boxed title to top left={xshift=1cm,yshift*=1mm-\tcboxedtitleheight}, varwidth boxed title*=-3cm,
	boxed title style={frame code={
					\path[fill=tcbcolback]
					([yshift=-1mm,xshift=-1mm]frame.north west)
					arc[start angle=0,end angle=180,radius=1mm]
					([yshift=-1mm,xshift=1mm]frame.north east)
					arc[start angle=180,end angle=0,radius=1mm];
					\path[left color=tcbcolback!60!black,right color=tcbcolback!60!black,
						middle color=tcbcolback!80!black]
					([xshift=-2mm]frame.north west) -- ([xshift=2mm]frame.north east)
					[rounded corners=1mm]-- ([xshift=1mm,yshift=-1mm]frame.north east)
					-- (frame.south east) -- (frame.south west)
					-- ([xshift=-1mm,yshift=-1mm]frame.north west)
					[sharp corners]-- cycle;
				},interior engine=empty,
		},
	fonttitle=\bfseries,
	title={#2},#1}{def}

%================================
% EXERCISE BOX
%================================

\makeatletter
\newtcbtheorem{exercise}{Exercise}{enhanced,
	breakable,
	colback=white,
	colframe=myb!80!black,
	attach boxed title to top left={yshift*=-\tcboxedtitleheight},
	fonttitle=\bfseries,
	title={#2},
	boxed title size=title,
	boxed title style={%
			sharp corners,
			rounded corners=northwest,
			colback=tcbcolframe,
			boxrule=0pt,
		},
	underlay boxed title={%
			\path[fill=tcbcolframe] (title.south west)--(title.south east)
			to[out=0, in=180] ([xshift=5mm]title.east)--
			(title.center-|frame.east)
			[rounded corners=\kvtcb@arc] |-
			(frame.north) -| cycle;
		},
	#1
}{def}
\makeatother

%================================
% SOLUTION BOX
%================================

\makeatletter
\newtcolorbox{solution}{enhanced,
	breakable,
	colback=white,
	colframe=myg!80!black,
	attach boxed title to top left={yshift*=-\tcboxedtitleheight},
	title=Solution,
	boxed title size=title,
	boxed title style={%
			sharp corners,
			rounded corners=northwest,
			colback=tcbcolframe,
			boxrule=0pt,
		},
	underlay boxed title={%
			\path[fill=tcbcolframe] (title.south west)--(title.south east)
			to[out=0, in=180] ([xshift=5mm]title.east)--
			(title.center-|frame.east)
			[rounded corners=\kvtcb@arc] |-
			(frame.north) -| cycle;
		},
}
\makeatother


%================================
% Exercise BOX
%================================

\makeatletter
\newtcbtheorem{exertion}{Exercise}{enhanced,
	breakable,
	colback=white,
	colframe=mygr,
	attach boxed title to top left={yshift*=-\tcboxedtitleheight},
	fonttitle=\bfseries,
	title={#2},
	boxed title size=title,
	boxed title style={%
			sharp corners,
			rounded corners=northwest,
			colback=tcbcolframe,
			boxrule=0pt,
		},
	underlay boxed title={%
			\path[fill=tcbcolframe] (title.south west)--(title.south east)
			to[out=0, in=180] ([xshift=5mm]title.east)--
			(title.center-|frame.east)
			[rounded corners=\kvtcb@arc] |-
			(frame.north) -| cycle;
		},
	#1
}{def}
\makeatother


%================================
% NOTE BOX
%================================

\usetikzlibrary{arrows,calc,shadows.blur}
\tcbuselibrary{skins}
\newtcolorbox{note}[1][]{%
	enhanced jigsaw,
	colback=gray!20!white,%
	colframe=gray!80!black,
	size=small,
	boxrule=1pt,
	title=\textbf{Note},
	halign title=flush center,
	coltitle=black,
	breakable,
	drop shadow=black!50!white,
	attach boxed title to top left={xshift=.5cm,yshift=-\tcboxedtitleheight/2,yshifttext=-\tcboxedtitleheight/2},
	minipage boxed title=1.5cm,
	boxed title style={%
			colback=white,
			size=fbox,
			boxrule=1pt,
			boxsep=2pt,
			underlay={%
					\coordinate (dotA) at ($(interior.west) + (-0.5pt,0)$);
					\coordinate (dotB) at ($(interior.east) + (0.5pt,0)$);
					\begin{scope}
						\clip (interior.north west) rectangle ([xshift=3ex]interior.east);
						\filldraw [white, blur shadow={shadow opacity=60, shadow yshift=-.75ex}, rounded corners=2pt] (interior.north west) rectangle (interior.south east);
					\end{scope}
					\begin{scope}[gray!80!black]
						\fill (dotA) circle (2pt);
						\fill (dotB) circle (2pt);
					\end{scope}
				},
		},
	#1,
}


%%%%%%%%%%%%%%%%%%%%%%%%%%%%%%
% SELF MADE COMMANDS
%%%%%%%%%%%%%%%%%%%%%%%%%%%%%%

\newcommand{\thm}[2]{\begin{Theorem}{#1}{}#2\end{Theorem}}
\newcommand{\cor}[2]{\begin{corolary}{#1}{}#2\end{corolary}}
\newcommand{\lem}[2]{\begin{lemma}{#1}{}#2\end{lemma}}
\newcommand{\wc}[2]{\begin{wconc}{#1}{}\setlength{\parindent}{1cm}#2\end{wconc}}
\newcommand{\thmcon}[1]{\begin{Theoremcon}{#1}\end{Theoremcon}}
\newcommand{\ex}[2]{\begin{Example}{#1}{}#2\end{Example}}
\newcommand{\dfn}[2]{\begin{Definition}[colbacktitle=green!60!black]{#1}{}#2\end{Definition}}
\newcommand{\dfnc}[2]{\begin{definition}[colbacktitle=green!60!black]{#1}{}#2\end{definition}}
\newcommand{\exer}[2]{\begin{exercise}{#1}{}#2\end{exercise}}
\newcommand{\pf}[2]{\begin{myproof}[#1]#2\end{myproof}}
\newcommand{\nt}[1]{\begin{note}#1\end{note}}
\newcommand{\rem}[1]{\begin{remark}{}{}#1\end{remark}}

\newcommand*\circled[1]{\tikz[baseline=(char.base)]{
		\node[shape=circle,draw,inner sep=1pt] (char) {#1};}}
\newcommand\getcurrentref[1]{%
	\ifnumequal{\value{#1}}{0}
	{??}
	{\the\value{#1}}%
}
\newcommand{\getCurrentSectionNumber}{\getcurrentref{section}}
\newenvironment{myproof}[1][\proofname]{%
        \vspace{-5pt}
	\proof[\bfseries #1: ]%
}{\endproof}
\newcounter{mylabelcounter}

\makeatletter
\newcommand{\setword}[2]{%
	\phantomsection
	#1\def\@currentlabel{\unexpanded{#1}}\label{#2}%
}
\makeatother

\tikzset{
	symbol/.style={
			draw=none,
			every to/.append style={
					edge node={node [sloped, allow upside down, auto=false]{$#1$}}}
		}
}


% =======================================================================
% TITLE FORMATS
% =======================================================================

\titleformat
{\section}
[display]
{\scshape}
{\color{blue} Lecture \thesection}
{0.5em}
{\color{blue} \centering \hrule \vspace{0.8em}}
[\vspace{0.8em} \hrule \vspace{1em}]

% =======================================================================
% TABLE OF CONTENTS
% =======================================================================

\renewcommand*\contentsname{Lectures}

%\makeatletter
%\renewcommand{\l@section}{\@dottedtocline{1}{1.5em}{2.6em}}
%\newcommand*\l@section{\@dottedtocline{1}{1.5em}{2.6em}}
%\makeatother

\makeatletter
\def\l@section{\@tocline{1}{5pt plus 2pt}{0pt}{2em}{\scshape}}% <- added

\def\@tocline#1#2#3#4#5#6#7{\relax
    \ifnum #1>-1% <- added
  \ifnum #1>\c@tocdepth % then omit
  \else
    \par \addpenalty\@secpenalty\addvspace{#2}%
    \begingroup \hyphenpenalty\@M
    \@ifempty{#4}{%
      \@tempdima\csname r@tocindent\number#1\endcsname\relax
    }{%
      \@tempdima#4\relax
    }%
    \parindent\z@ \leftskip#3\relax \advance\leftskip\@tempdima\relax
    \rightskip\@pnumwidth plus4em \parfillskip-\@pnumwidth
    #5\leavevmode\hskip-\@tempdima #6\nobreak\relax
    \hfil\hbox to\@pnumwidth{\@tocpagenum{#7}}\par
    \nobreak
    \endgroup
  \fi% <- added
\fi}

\makeatother

\begin{document}
\includepdf{cover}

These are lecture notes for the course MTH402: Functional Analysis taught by Chandrakant Aribam during the monsoon session of 2022. I live-\TeX{}-ed them on Emacs. Please report bugs/errors, if any.

\hfill Ronald Huidrom

\tableofcontents

\section{The Course Commences, 22/07/2022} 

The lecture starts with a short review of elementary ideas of vector spaces. Morphisms are structure preserving maps. An endomorphism on a mathematical structure $S$ is a structure preserving map from $S$ to itself.

\dfn{Linear endomorphism}{Let $V$ be a vector space. A linear endomorphism is a linear transformation $T: V \to V$. }

Set of all linear endomorphisms on a vector space $V$ over a field $F$ is denoted by End$_F(V)$. An automorphism on $V$ is an endomorphism on $V$ which is also an isomorphism (one-one and onto).

Recall that every field is also a vector space over itself. Thus, we may define a linear map from a vector space to a field. Indeed, we have a special name for this map.

\dfn{Linear functional}{Let $V$ be a vector space. Let $F$ be a field. Then a linear functional is a linear transformation $T: V \to F$.}

The following are examples of linear functionals.

\begin{enumerate}
\item A linear map $T: \bbR \to \bbR $ defined by $T(x) = \alpha x$ for some $\alpha \in \bbR$. 
\item A linear map $T: \bbR^2 \to \bbR$ defined by $T(x_1, x_2) = \beta x_1$ for some $\beta \in \bbR$. 
\item A linear map $\pi: \bbR^2 \to \bbR^2$ defined by $\pi(x_1, x_2) = x_1$. 
\item Let $T_1:V_1 \to V_1$ be a linear map and $T_2: V_1 \to F$ be a linear functional. Then $T_2\circ T_1: V_1 \to F$ is a linear functional. 
\end{enumerate}

Consider the following ODE: 
\begin{equation}
\label{eq:1}
a_n \frac{d^ny}{dx^n} + a_{n-1} \frac{d^{n-1}y}{dx^{n-1}} + \cdots + a_1 \frac{dy}{dx} + a_0 = 0.
\end{equation}

Let $a_0=0$. If $y_1(x)$ and $y_2(x)$ are two solutions, then $(y_1+y_2)(x)$ is also a solution. Also $\alpha y_1(x)$ is also a solution for some $\alpha \in \bbR$. Let $S$ be the set of all solutions when $a_0 = 0$. Clearly $S$ is a vector space over $\bbR$.

We now define the operator 
\begin{displaymath}
L \coloneqq a_n \frac{d^n}{dx^n} + a_{n-1} \frac{d^{n-1}}{dx^{n-1}} + \cdots + a_1 \frac{d}{dx}
\end{displaymath}

which operates on the vector space $C^n(\bbR)\coloneqq \{f:\bbR \to \bbR | f \text{ is differentiable on } \bbR \text{ upto } n \text{ times}\}$ over $\bbR$. Let $X$ be the set of all functions on $\bbR$. Then $X$ is a vector space (also called function space) and $L:C^n(\bbR) \to X$ is a linear map. Clearly $S = \ker(L)$.

\exer{}{What is the dimension of the vector space $L$ defined above?}

\section{Say Hello to Banach Spaces, 23/07/2022}

A vector space is also called a linear space in that it makes sense to form linear combinations. Defining a norm on a linear space makes it a metric space. Then it makes sense to talk about continuous maps between such linear spaces. Banach spaces are linear spaces with a norm such that the metric space generated is also complete.

Let $X\subset \bbR$. When is $X$ compact? From analysis, we know $X$ is compact if and only if it is closed and bounded. We define 
\begin{displaymath}
C(X) = \{ f: X \to \bbR \mid f \text{ is continuous on } X\}.
\end{displaymath}

When is $C(X)$ compact? To answer this question, we need to ask: what is the topology on $C(X)$? Suppose $X= [0,1]$. What if, the topology is the one induced by the metric space generated by the norm 
\begin{displaymath}
\Vert h \Vert_{\infty} = \sup\{\Vert h(x) \Vert \mid x \in X\}.
\end{displaymath}

\dfn{Normed linear space}{Let $N$ be a vector space over $\bbR$ (or $\bbC$). Then $N$ is a normed linear space if $\exists a$ function $\Vert \cdot \Vert : N \to \bbR$ such that 
\begin{enumerate}
\item\label{item:1} $\Vert x \Vert \geq 0$ and $\Vert x \Vert = 0 \iff x = 0$. 
\item\label{item:2} $\Vert x+y \Vert \leq \Vert x \Vert + \Vert y \Vert$. 
\item\label{item:3} $\Vert \alpha x \Vert = |\alpha| \Vert x \Vert$ for all $\alpha \in \bbR \text{ (or }\alpha \in \bbC), x \in N$.
\end{enumerate}}

\label{exer-2}\exer{Norm induces a metric.}{
\begin{enumerate}
\item\label{item:4} Show that $d(x,y) = \Vert x-y \Vert$ is a metric on $N$. 
\item\label{item:7} Since $N$ is a metric space, we can talk about Cauchy sequences in $N$. Let $\{x_n\}$ be a sequence in $N$. Show that $\{x_n\}$ is Cauchy if and only if $\Vert x_n-x_m \Vert \to 0$ as $n, m \to 0$. 
\item\label{item:8} Show that if $\{x_n\}$ converges to $x$, then $\Vert x_n \Vert$ converges to $\Vert x \Vert$. 
\item\label{item:9} Show that $\left| \Vert x \Vert - \Vert y \Vert \right| \leq \Vert x - y \Vert$ for all $x,y \in N$. 
\item\label{item:10} Show that if $\{x_n\}$ is Cauchy, then $\{ \Vert x_n \Vert\}$ is also Cauchy. 
\item\label{item:11} Suppose $x_n \to x$ and $y_n \to y$. Show that $x_n+y_n \to x+y$. 
\item\label{item:12} Suppose $\alpha_n \in \bbR$, $\alpha_n \to \alpha$ and $x_n \to x$. Show that $\alpha_nx_n = \alpha x$.
\end{enumerate}
}

\nt{See that (\ref{item:8}) of Exercise \ref{exer-2} shows the norm on $N$ is continuous.}

We recall that a complete metric space is one in which every Cauchy sequence in the space converges to a point in the space itself. Keeping this in mind, we are well-equiped to define a Banach space.

\dfn{Banach space}{A complete normed linear space $N$ is called a Banach space.}

We look at some examples of Banach spaces:
\begin{enumerate}
\item\label{item:13} $\bbR$, $\bbC$ are Banach spaces with their usual metric.
\item\label{item:14} $\bbR^n$ is a Banach space with either of the norms 
\begin{displaymath}
\Vert (x_1, \ldots, x_n) \Vert = |x_1| + \cdots + |x_n| \hspace*{1cm}\text{ or }\hspace*{1cm} \Vert (x_1, \ldots, x_n) \Vert = \left( \sum_{i=1}^n |x_i|^2 \right)^{1/2}.
\end{displaymath} 
\item\label{item:15} $l_p^n$-spaces are Banach spaces with the $p$-norm 
\begin{displaymath}
\Vert (x_1, \ldots, x_n) \Vert = \left( \sum_{i=1}^n |x_i|^p \right)^{1/p}.
\end{displaymath}
\end{enumerate}

\exer{}{Prove the triangle inequality for $p$-norm.}

The scheduled class on 25/08/2022 was dismissed.

\section{The Misery of Banach Spaces, 26/08/2022}

Normed linear spaces can be infinite dimensional over $\bbR$ or $\bbC$. Recall that Banach spaces are complete normed linear spaces. The linear space $(\bbQ, |\cdot|)$ with the usual metric is not complete.

\exer{}{Can you think of a vector spaces over $\bbR$ which is not complete with respect to some norm?}

Consider $l_p^n$ with $1\leq p < \infty$. Is this complete? Suppose $x_m$, $x_{m^{\prime}}\in l_p^n$. Let $\epsilon > 0$ be any fixed real number. Suppose for sufficiently large $m, m^{\prime}$ sufficiently large,
\begin{align*}
  \Vert x_m - x_{m^{\prime}} \Vert &< \epsilon \\
  \sum_{}^{} |x_{m_i}-x_{m^{\prime}_i}|^p &< \epsilon^p \\
  |x_{m_i}-x_{m^{\prime}_i} |^p \leq \sum_{}^{} |x_{m_i}-x_{m^{\prime}_i}|^p &< \epsilon^p \\
  |x_{m_i}-x_{m^{\prime}_i} | &< \epsilon.
\end{align*}
That is, $\{x_{m,i}\}_m$ is Cauchy in $\bbR$. $\{x_{m,1}\}_m$ converges in $\bbR$, say $\displaystyle \lim_{m\to\infty} x_{m,i} = a_i$.

We put $a=(a_1, \cdots, a_n)$. Now, $|x_{m,i}-a_i| < \epsilon/n^{1/p}$ for $m \geq N_i$. For $N_0 = \max\{N_1, \cdots, N_n\}$, we have,
\begin{align*}
|x_m-a_i| < \epsilon/n^{1/p} \hspace*{1cm}\text{ for all $m>N_0$ and for all $i$}.
\end{align*}
Then 
\begin{align*}
  \sum_{i=1}^m |x_m-a_i|^p &< \epsilon^p \\
  \Vert x_m-a \Vert = \sum_{i=1}^n |x_{m,i}-a_i | &< \epsilon,
\end{align*}
for all $m > N_0$. Therefore, $\displaystyle \lim_{m\to\infty} x_m = a$ so that $l_p^n$ is a Banach space.

We represent $l_p^{\infty}$ as $l_p$. It is an infinite dimensional linear space such that
\begin{displaymath}
\sum_{i=1}^{\infty} |x_i|^p < \infty.
\end{displaymath}
Taking square root yields its norm, which is 
\begin{equation}
\label{eq:2}
\Vert x \Vert _p = \left( \sum_{n=1}^{\infty} |x_n|^p \right)^{1/p}.
\end{equation}

\exer{Triangle inequality}{Show that the norm defined above satisfies triangle inequality.}

We now introduce $L_p$ space as a space of functions $f: \bbR \to \bbR$ equipped with the norm 
\begin{equation}
\label{eq:3}
\Vert f \Vert_p = \left( \int |f(x)|^p dm(x) \right)^{1/p},
\end{equation}
where $m$ is a measure on $\bbR$. We may also write 
\begin{displaymath}
L_p = \{ f: \bbR \to \bbR \mid \int_{} |f(x)|^p dm(x) < \infty \}.
\end{displaymath}

\section{29/08/2022}

Continuous functions on compact sets are always bounded. In fact, continuous functions maps compact sets to compact sets.

\thm{}{Let $N$ be a normed linear space and $M \subset N$ which is a closed linear subspace. Then 
\begin{enumerate}
\item\label{item:5} Consider the quotient $N/M = \{ x + M \mid x \in N \text{ and } x_1+M = x_2 + M \text{ iff } x_1-x_2 \in M \}$. Define
\begin{displaymath}
\Vert x + M \Vert = \inf \{\Vert x + m \Vert \mid m \in M \},
\end{displaymath}
then $N/M$ is a linear subspace over $\bbR$. 
\item\label{item:6} If $N$ is Banach, then $N/M$ is Banach.
\end{enumerate}}

\begin{proof}
\label{sec:29082022}
\begin{enumerate}
\item\label{item:16} $\Vert x + M \Vert = \inf \{ \Vert x + m \Vert \mid m\in M \} \implies \Vert x + M \Vert \geq 0$. Suppose $\Vert x + M \Vert = 0$. Then $\exists$ a sequence $\{ t_k \} \subset M$ such that $\Vert x + t_k \Vert \to 0$. That is, $0 < 1/k \implies \exists \Vert x + t_k \Vert < 1/k$. As $M$ is closed, there exists a subsequence $\{t_k\} \subset M$ such that $\{t_k\}$ is convergent and let $\displaystyle t = \lim_{r\to\infty} t_{k_r}$. Then $t\in M$ as $M$ is closed. Since $t_{n_r} \to t$, we have $x+t_{n_r} \to x + t$. Then $\Vert x + t_{n_r} \Vert \to \Vert x + t \Vert$ so that $\Vert x+t \Vert = 0$, from which we get $t = -x$. Therefore, $x+M = 0$.

  Now, we want to prove triangle inequality. See that, for $m=u+v \in M$, we have 
\begin{align*}
  \Vert x + y + m \Vert \leq \Vert x + u \Vert + \Vert y + v \Vert \\
  \inf \Vert x + y + m \Vert \leq \Vert x + y + m \Vert \leq \Vert x + u \Vert + \Vert y + v \Vert
\end{align*}
Then 
\begin{align*}
  \inf_{u\in M} \Vert x + u \Vert \leq \Vert x + u \Vert \\
  \left(\inf_{u\in M} \Vert x + u \Vert \right) + \Vert y + v \Vert \leq \Vert x + u \Vert + \Vert y + v \Vert \\
  \inf_{m\inM} \Vert x + y + m \Vert \leq \inf_{u\in M } \Vert x + u \Vert + \Vert y + b \Vert.
\end{align*}
Take $\inf$ on $v\in M$. Then we are done. 
\item\label{item:18} Let $\{ x_n + M\}_n$ be a Cauchy sequence in $N/M$. That is, $\Vert x_n + M - x_m + M\Vert \to 0$ as $n, m \to \infty$. We see that 
\begin{align*}
  \Vert x_{n_1} + M - x_{n_2} + M \Vert \frac{1}{2} \implies \Vert y_1-y_2 \Vert < \frac{1}{2} \\
  \Vert x_{n_2} + M - x_{n_3} + M \Vert \frac{1}{2^2} \implies \Vert y_2-y_3 \Vert < \frac{1}{2^2} \\
  \cdots\\
  \Vert x_{n_k} + M - x_{n_{k+1}} + M \Vert < \frac{1}{2^k},
\end{align*}
and so on.
\end{enumerate}
\end{proof}


\nt{In the above, defining $\Vert x + M \Vert = \Vert x \Vert$ does not work. Defining $\Vert x + M \Vert = \sup \{ \Vert x + m \Vert \mid m \in M \}$ will not work either.} 

\end{document}